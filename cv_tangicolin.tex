% Taille de la police du CV et le format
\documentclass[17pt,a4paper]{cv}

% Un certain nombre de packages sont déjà inclus par la classe CV
\usepackage[french]{babel}
\usepackage[utf8]{inputenc}
\usepackage[T1]{fontenc}

% Langue
\Language{francais}

% Méta-données PDF
\hypersetup{
	pdfauthor   = {Tangi COLIN	},		% C'est vous qui avez fait le PDF
	pdftitle    = {Curriculum Vitae},
%	pdfsubject  = {},
%	pdfkeywords = {},
	pdfcreator  = {PDFLaTeX},						% Vous l'avez fait avec LATeX !
	pdfproducer = {PDFLaTeX}
}

% Couleur des liens hypertextes, si vous voulez les faire ressortir
\definecolor{urlcolor}{rgb}{0, 0, 0.5}

% Version (0 = simple, devrait rentrer sur une page et 1 = détaillé)
% Par défaut, tout est inclus dans la version simple sauf les cours (\Course{}).
% Pour n'inclure un élément que dans la version détaillée, ajouter l'argument [\detailedOnly]
% (cela n'est toutefois pas implémenté partout, votre nom par exemple ne peut pas ne pas figurer dans la version simple)
\Detailed{0}

% Marges
\geometry{
	hmargin=1.1cm,			% Marges gauche-droite
	vmargin=1.5cm				% Marges haut-bas
}

% Largeur de la colonne de gauche
\setlength{\colwidth}{3,2cm}

% Espacements
\smallskipamount=0.4cm	% Espace entre les différents entrées d'une section
\medskipamount=0.6cm		% Espace entre les sections
\bigskipamount=1cm		% Espace entre le chapeau et l'objectif ou la



%%%%%%%%%%% commande modernfont %%%%%%%%%%%
\newcommand{\modernfont}{%
\fontencoding{\encodingdefault}%
\fontfamily{pag}%
\fontseries{m}%
\fontshape{n}%
\selectfont}


\begin{document}

% Le CV commence ici
%%%%%%%%%%%%%%%%%%%%

% Chapeau (= informations personnelles)
\begin{heading}
 	\Name{\textbf{Tangi COLIN}}
	\Address{10, les Tonnelles, 38420 Le Versoud, France}
	\Telephone{~06~**}
	\Email{tangicolin@gmail.com} % Un lien
                        %%hypertexte est automatiquement créé pour l'Email
	\Nationality{Français}
	\DateOfBirth{17 Octobre 1990}
	\Age{}
	\Gender{M}	% M = Masculin, F = Féminin. Est utilisé pour savoir s'il faut écrire Né ou Née par exemple ;-)
	%\MaritalStatus{\underline{Statu} Célibataire}
	\Mobility{\underline{Mobilité :} Permis B + véhicule}
	%\Photo{}	% Vous pouvez choisir de mettre une photo ou pas.
	% sa dimension est sans importance, au final sa hauteur est fixée à 3cm
	% et son rapport hauteur/largeur est conserve.
	% Meilleure est sa résolution, mieux c'est !
\end{heading}

% Objectif (optionnel)
%\begin{objective}
%\end{objective}
	\begin{center}\modernfont\huge{Ingénieur en Systèmes
embarqués Linux\newline}
	\end{center}
\begin{section}{\modernfont\Large{Expériences professionnelles:}}
	\begin{entry}
		\Date{10/2016 - 12/2016}
		\Type{CDI}
		\Place{BH-Technologies}
		\Locality{38000 Grenoble}
		\Country{FRANCE}
		\Activity{\large{Développeur au sein de l'équipe firmware, mise en place d' une
intégration continue avec tests automatisés sur cible}}
	\end{entry}
	\begin{entry}
		\Date{07/2013 - 09/2016}
		\Type{CDI}
		\Place{Witekio}
		\Locality{69130 Ecully}
		\Country{FRANCE}
		\Activity{\large{Réalisation de nombreux projets concernant le
développement de drivers Linux et Android sur des plates-formes
embarquées ARM cortex-A orientés multimédia tels que les
architectures TI OMAP et NXP IMX6. Cela inclus entre autres le développement
et bugfix de drivers camera,wifi, customisation de BSP Linux ou Android,
optimisation du temps de boot, power management, secure boot, firmware upgrade et le développement d'un framework
de test pour des BSP Linux/Android ...}
\newline\newline
\Huge{.}\large{ 04/2014 - 08/2015: Projet d'infrastructure
d'intégration continue "as a Service" basée sur les technologies Mesos, Docker,
Jenkins (jobDSL) et Gerrit pour le compte de Schneider. }
 \newline\newline
\Huge{.}\large{ 09/2015 - 03/2016: Expert Linux pour le projet de
plates-formes embarquées du groupe Schneider}}
	\end{entry}
	\begin{entry}
		\Date{02/2013 - 06/2013}
		\Type{Stage de fin d'étude}
		\Place{Witekio}
		\Locality{69130 Ecully}
		\Country{FRANCE}
		\Activity{\large{Étude des différentes techniques de debugging,
profiling et tracing du noyau linux.
(tracepoint, kprobes, ftrace, perf, systemtap, lttng...)}}
	\end{entry}
	\begin{entry}
		\Date{01/2012 - 07/2012}
		\Type{Projet Industriel + CDD (1 mois)}
		\Place{Trixell}
		\Locality{38430 Moirans}
		\Country{FRANCE}
		\Activity{\large{Intégration de Web Service ReST et SOAP sur système
embarqué Xilinx virtex 4}}
	\end{entry}
\end{section}


\begin{section}{\modernfont\Large{Études et Diplômes:}}
	\begin{entry}
		\Date{2013}
		\Type{}
		\Activity{\large{Diplôme d'ingénieur en Électronique
Informatique et Systèmes Embarqués}}
		\Place{Grenoble INP École Supérieur d'Ingénieur en Systèmes Avancés et
Réseaux}
		\Locality{Valence}
		\Country{FRANCE}
	\end{entry}
	\begin{entry}
		\Date{2008}
		\Type{}
		\Place{Lycée Jules Algoud}
		\Locality{Valence}
		\Country{FRANCE}
		\Activity{\large{BAC S  (spécialité science de l'ingénieur) avec
mention}}
	\end{entry}
\end{section}

\newpage
\begin{section}{\modernfont\Large{Langues et Compétences maîtrisées:}}
\singleEntry{\large{$\cdot$ Anglais (anglais technique parlé couramment)}}
	\singleEntry{\large{$\cdot$ Langages informatiques : C,Shell/Bash, Python, Java/Groovy}}
	\singleEntry{\large{$\cdot$ Protocoles réseaux : IPv4/IPv6, OSI,TCP,UDP, SSH, HTTP(Rest), LPWAN (Sigfox et Lora), COAP, BGP...}}
  \singleEntry{\large{$\cdot$ Outils IT : Docker, Mesos, Ansible}}
  \singleEntry{\large{$\cdot$ Outils debug noyau : Ftrace, Perf, LTTng, BPF}}
  \singleEntry{\large{$\cdot$ Intégration continue : Jenkins, Gitlab Runners}}
  \singleEntry{\large{$\cdot$ Système de contrôle de version  : Git, Gerrit, Gitlab, Bitbucket}}
  \singleEntry{\large{$\cdot$ Outils de complitation : Yocto, Buildroot, Android.mk, Cmake}}
  \singleEntry{\large{$\cdot$ Méthodologie de test : développement piloté par les tests (TDD), développement piloté par le comportement (BDD)}}
  \singleEntry{\large{$\cdot$ Expertise en conception des
architectures matérielles/logicielles des systèmes embarqués et en architecture micro-service}}
  \singleEntry{\large{$\cdot$ Expertise en administration des systèmes linux}}
  \singleEntry{\large{$\cdot$ Très bonne connaissance générale en informatique (pratique une veille technique journalière)}}
  \singleEntry{\large{$\cdot$ Formé a la méthodologie Agile (Scrum)}}
  \smallSkip
\end{section}

\end{document}
